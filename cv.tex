%-----------------------------------------------------------------------------------------------------------------------------------------------%
%	The MIT License (MIT)
%
%	Copyright (c) 2021 Jitin Nair
%
%	Permission is hereby granted, free of charge, to any person obtaining a copy
%	of this software and associated documentation files (the "Software"), to deal
%	in the Software without restriction, including without limitation the rights
%	to use, copy, modify, merge, publish, distribute, sublicense, and/or sell
%	copies of the Software, and to permit persons to whom the Software is
%	furnished to do so, subject to the following conditions:
%	
%	THE SOFTWARE IS PROVIDED "AS IS", WITHOUT WARRANTY OF ANY KIND, EXPRESS OR
%	IMPLIED, INCLUDING BUT NOT LIMITED TO THE WARRANTIES OF MERCHANTABILITY,
%	FITNESS FOR A PARTICULAR PURPOSE AND NONINFRINGEMENT. IN NO EVENT SHALL THE
%	AUTHORS OR COPYRIGHT HOLDERS BE LIABLE FOR ANY CLAIM, DAMAGES OR OTHER
%	LIABILITY, WHETHER IN AN ACTION OF CONTRACT, TORT OR OTHERWISE, ARISING FROM,
%	OUT OF OR IN CONNECTION WITH THE SOFTWARE OR THE USE OR OTHER DEALINGS IN
%	THE SOFTWARE.
%	
%
%-----------------------------------------------------------------------------------------------------------------------------------------------%

%----------------------------------------------------------------------------------------
%	DOCUMENT DEFINITION
%----------------------------------------------------------------------------------------

% article class because we want to fully customize the page and not use a cv template
\documentclass[a4paper,12pt]{article}

%----------------------------------------------------------------------------------------
%	FONT
%----------------------------------------------------------------------------------------

% % fontspec allows you to use TTF/OTF fonts directly
% \usepackage{fontspec}
% \defaultfontfeatures{Ligatures=TeX}

% % modified for ShareLaTeX use
% \setmainfont[
% SmallCapsFont = Fontin-SmallCaps.otf,
% BoldFont = Fontin-Bold.otf,
% ItalicFont = Fontin-Italic.otf
% ]
% {Fontin.otf}

%----------------------------------------------------------------------------------------
%	PACKAGES
%----------------------------------------------------------------------------------------
\usepackage{url}
\usepackage{parskip} 	
\usepackage[bitstream-charter]{mathdesign}
\usepackage[T1]{fontenc}
%other packages for formatting
\RequirePackage{color}
\RequirePackage{graphicx}
\usepackage[usenames,dvipsnames]{xcolor}
\usepackage[scale=1]{geometry}
 \geometry{
 a4paper,
 total={170mm,257mm},
 left=20mm,
 right=20mm,
 top=30mm,
 }
%tabularx environment
\usepackage{tabularx}

%for lists within experience section
\usepackage{enumitem}

% centered version of 'X' col. type
\newcolumntype{C}{>{\centering\arraybackslash}X} 

%to prevent spillover of tabular into next pages
\usepackage{supertabular}
\usepackage{tabularx}
\newlength{\fullcollw}
\setlength{\fullcollw}{0.47\textwidth}

%custom \section
\usepackage{titlesec}				
\usepackage{multicol}
\usepackage{multirow}

%CV Sections inspired by: 
%http://stefano.italians.nl/archives/26
\titleformat{\section}{\Large\scshape\raggedright}{}{0em}{}[\titlerule]
\titlespacing{\section}{0pt}{10pt}{10pt}

%for publications
\usepackage[style=authoryear,sorting=ynt, maxbibnames=2]{biblatex}

%Setup hyperref package, and colours for links
\usepackage[unicode, draft=false]{hyperref}
\definecolor{linkcolour}{rgb}{0,0.2,0.6}
\hypersetup{colorlinks,breaklinks,urlcolor=linkcolour,linkcolor=linkcolour}
\addbibresource{citations.bib}
\setlength\bibitemsep{1em}

%for social icons
\usepackage{fontawesome5}

%debug page outer frames
%\usepackage{showframe}

%----------------------------------------------------------------------------------------
%	BEGIN DOCUMENT
%----------------------------------------------------------------------------------------
\begin{document}

% non-numbered pages
\pagestyle{empty} 

%----------------------------------------------------------------------------------------
%	TITLE
%----------------------------------------------------------------------------------------

% \begin{tabularx}{\linewidth}{ @{}X X@{} }
% \huge{Your Name}\vspace{2pt} & \hfill \emoji{incoming-envelope} email@email.com \\
% \raisebox{-0.05\height}\faGithub\ username \ | \
% \raisebox{-0.00\height}\faLinkedin\ username \ | \ \raisebox{-0.05\height}\faGlobe \ mysite.com  & \hfill \emoji{calling} number
% \end{tabularx}

\begin{tabularx}{\linewidth}{@{} C @{}}
\Huge{Jorge Luis Alvarado} \\[7.5pt]
\href{https://github.com/jorge-alvarado-revata}{\raisebox{-0.05\height}\faGithub\ jorge-alvarado-revata} \ $|$ \ 
\href{mailto:j.revatta@gmail.com}{\raisebox{-0.05\height}\faEnvelope \ j.revatta@gmail.com} \ $|$ \
\href{tel:+51926137221}{\raisebox{-0.05\height}\faMobile \ +51-926-137-221} \\
\end{tabularx}

%----------------------------------------------------------------------------------------
% EXPERIENCE SECTIONS
%----------------------------------------------------------------------------------------

%Interests/ Keywords/ Summary
\section{Resumen}
Ingeniero de Sistemas (UNMSM) con experiencia en roles Developer, Senior Developer, Líder técnico en tecnología Web, APIs y aplicaciones transaccionales en banca privada y sector público. Experto en .NET, implementación de consumo de servicios REST, SOAP, autenticación OAUTH e integración de seguridad cloud. Experiencia en Angular hasta versiones actuales. Experiencia en Java Spring Boot y Flutter. Docente en Python y Angular.

\section{Objetivos}
Desarrollar componentes y APIs de gran calidad, con código legible, ordenado y siguiendo mejores prácticas de programación, patrones y principios SOLID. Lograr la satisfacción del cliente atendiendo pro-activamente las necesidades del proyecto.
%Experience
\section{Experiencia}

\textsl{FRONTEND DEVELOPER}\\[3.75pt]
\begin{tabularx}{\linewidth}{ @{}l r@{} }
\textsl{INETUM, PERÚ – Contraloría General del Perú} & \hfill Octubre 2023 - Febrero 2024 \\[3.75pt]
\multicolumn{2}{@{}X@{}}{
\begin{minipage}[t]{\linewidth}
    \begin{itemize}[nosep,after=\strut, leftmargin=1em, itemsep=3pt]
        \item[--] Implemente mejoras en aplicativo generador de formularios encargado de nuevos componentes de datasource.
        \item[--] Implemente de microservicios en Java 17 para mejoras de datos y consultas de servicios SOAP de seguridad.
        \item[--] Implemente microservicios para gestión de configuración de plantillas y tablas de procesos con ORACLE Database.
        \item[--] Implementé esquema de validación de parametros de entrada a los APIs.
    \end{itemize}
\end{minipage}
}
\end{tabularx}

\textsl{SENIOR DEVELOPER}\\[3.75pt]
\begin{tabularx}{\linewidth}{ @{}l r@{} }
\textsl{MINEDU, PERÚ – CONCURSO DOCENTE 2023} & \hfill Mayo 2023 - Junio 2023 \\[3.75pt]
\multicolumn{2}{@{}X@{}}{
\begin{minipage}[t]{\linewidth}
    \begin{itemize}[nosep,after=\strut, leftmargin=1em, itemsep=3pt]
        \item[--] Implemente exitosamente nuevos microservicios C\# con esquemas de validación.
        \item[--] Optimización de aplicativo Flutter.
        \item[--] Microservicios para gestión de configuración de plantillas y tablas de procesos.
        \item[--] Optimización de consumo de servicios RabbitMQ.
    \end{itemize}
\end{minipage}
}
\end{tabularx}

\textsl{SENIOR DEVELOPER}\\[3.75pt]
\begin{tabularx}{\linewidth}{ @{}l r@{} }
\textsl{MINEDU, PERÚ – PROYECTO SAISA} & \hfill Julio 2022 - Marzo 2023 \\[3.75pt]
\multicolumn{2}{@{}X@{}}{
\begin{minipage}[t]{\linewidth}
    \begin{itemize}[nosep,after=\strut, leftmargin=1em, itemsep=3pt]
        \item[--] Implemente de componentes reutilizables en ReactiveForm, integración con servicios BackEnd con Microservicios C\# y base de datos SQL Server.
        \item[--] Implemente componentes BackEnd MinIO S3 para carga y descarga de documentos pdf.
        \item[--] Implemente componentes de reportes PDF.
    \end{itemize}
\end{minipage}
}
\end{tabularx}

\newpage

\textsl{SENIOR DEVELOPER}\\[3.75pt]
\begin{tabularx}{\linewidth}{ @{}l r@{} }
\textsl{INTELEGO SCRL, PERÚ – GESTOR DE COMPRAS CLOUD} & \hfill Febrero 2021 - Mayo 2022 \\[3.75pt]
\multicolumn{2}{@{}X@{}}{
\begin{minipage}[t]{\linewidth}
    \begin{itemize}[nosep,after=\strut, leftmargin=1em, itemsep=3pt]
        \item[--] Implemente un sistema de gestión de compras para restaurantes y servicios de catering en Angular 9, Python Django y Flutter 2, implementación de uso de sistemas de autenticación cloud con Firebase.
        \item[--] Implemente carga de imágenes en servicios cloud.
        \item[--] Implemente componentes de tablas y formularios de carga y visualización.
    \end{itemize}
\end{minipage}
}
\end{tabularx}

\textsl{DOCENTE}\\[3.75pt]
\begin{tabularx}{\linewidth}{ @{}l r@{} }
\textsl{UTEC, PERÚ – DOCENTE CURSOS PRESENCIAL} & \hfill Febrero 2017 - Enero 2020 \\[3.75pt]
\multicolumn{2}{@{}X@{}}{
\begin{minipage}[t]{\linewidth}
    \begin{itemize}[nosep,after=\strut, leftmargin=1em, itemsep=3pt]
        \item[--] Profesor de programación en Python y C++.
        \item[--] Profesor de curso de Servicios empresariales.

    \end{itemize}
\end{minipage}
}
\end{tabularx}

\textsl{CONSULTOR}\\[3.75pt]
\begin{tabularx}{\linewidth}{ @{}l r@{} }
\textsl{OP ENERGETICA SAC, PERÚ – CONSULTORIA SISTEMAS} & \hfill Junio 2014 - MAyo 2016 \\[3.75pt]
\multicolumn{2}{@{}X@{}}{
\begin{minipage}[t]{\linewidth}
    \begin{itemize}[nosep,after=\strut, leftmargin=1em, itemsep=3pt]
        \item[--] Diseño y desarrollo de sistema de carga laboral obreros en Jquery, Bootstrap, C\#.
        \item[--] Diseño y desarrollo de gestor de planificación de emergencias en Javascript, Bootstrap, Python.

    \end{itemize}
\end{minipage}
}
\end{tabularx}


\textsl{LÍDER DE EQUIPO}\\[3.75pt]
\begin{tabularx}{\linewidth}{ @{}l r@{} }
\textsl{NTT DATA, PERÚ – Jefatura y Líder técnico} & \hfill Abril 2012 - Julio 2013 \\[3.75pt]
\multicolumn{2}{@{}X@{}}{
\begin{minipage}[t]{\linewidth}
    \begin{itemize}[nosep,after=\strut, leftmargin=1em, itemsep=3pt]
        \item[--] Estimación, seguimiento de proyectos, participación en sesiones de diseño técnico y supervisión y apoyo a equipo de 10 analistas programadores.

    \end{itemize}
\end{minipage}
}
\end{tabularx}

\textsl{ANALISTA DESARROLLADOR}\\[3.75pt]
\begin{tabularx}{\linewidth}{ @{}l r@{} }
\textsl{BCP, PERÚ – ANÁLISIS Y DISEÑO} & \hfill Enero 2003 - Marzo 2012 \\[3.75pt]
\multicolumn{2}{@{}X@{}}{
\begin{minipage}[t]{\linewidth}
    \begin{itemize}[nosep,after=\strut, leftmargin=1em, itemsep=3pt]
        \item[--] Liderazgo técnico de equipo de 5 programadores para proyectos Banca Móvil y Gestor de inversiones .
        \item[--] Análisis y Diseño de aplicaciones en equipo HomeBanking.
        \item[--] Mantenimiento y pase a producción de aplicaciones Web para clientes de la banca personal.

    \end{itemize}
\end{minipage}
}
\end{tabularx}
%----------------------------------------------------------------------------------------
%	EDUCATION
%----------------------------------------------------------------------------------------
\section{EDUCACIÓN}
\begin{tabularx}{\linewidth}{@{}l X@{}}	
2023 - present & Maestría Estadística Matemática \hfill \textsl{UNMSM} \\

2009 - 2010 & Máster en I+D+i \hfill \textsl{UNIA} \\ 

1997 - 2003 & Ingeniería de Sistemas \hfill \textsl{UNMSM} \\

1997 - 2003 & Ingles Intermedio B2 \hfill \textsl{Peruano Británico} \\
\end{tabularx}
%Projects


\section{CERTIFICADOS}

\begin{tabularx}{\linewidth}{@{}l X@{}}	
ENERO 2023 & Building AI \hfill \textsl{MinnaLearn \& University of Helsinki} \\

OCT 2021 & AWS Machine Learning Foundation \hfill \textsl{Udacity} \\ 

SEP 2020 & Modern JavaScript ES6  \hfill \textsl{Coursera} \\

AGO 2020 & Google Cloud Fundamentals: Core Infrastructure  \hfill \textsl{Coursera} \\

ENERO 2019 & Elements of AI  \hfill \textsl{University of Helsinki \& Reacktor} \\

MAYO 2004 & Microsoft Certified Solution Developer  \hfill \textsl{Microsoft} \\

\end{tabularx}

%----------------------------------------------------------------------------------------
%	SKILLS
%----------------------------------------------------------------------------------------
\section{HABILIDADES}
\begin{tabularx}{\linewidth}{@{}l X@{}}
Front &  \normalsize{TypeScript, Angular 17, Jasmine, NgRX}\\
Mobile & \normalsize{Flutter, LocalStorage, SQLite } \\
UI  &  \normalsize{Material Design, Bootstrap, TailWind, CSS}\\  
Framework  &  \normalsize{Django, ASP.NET, .NET Core, Spring Boot, DRF, FastAPI, Swagger}\\
Language  &  \normalsize{Python, C\#, Java, C++, T-SQL, PL/SQL }\\
Cloud  &  \normalsize{Google Cloud, AWS, Firebase Auth, Firebase storage}\\
BD  &  \normalsize{SQL Server, Oracle Database, PostgreSQL }\\
\end{tabularx}



\vfill
\center{\footnotesize Última versión: \today}

\end{document}
